\documentclass[12pt,a4paper]{article}

\usepackage[utf8]{inputenc}
\usepackage{amsmath}
\usepackage{amsfonts}
\usepackage{makeidx}
\usepackage{amssymb}
\usepackage{hyperref}
\usepackage{graphicx}
\usepackage{float}
\usepackage{indentfirst}
\usepackage[spanish,es-tabla]{babel}
\restylefloat{table}



\title{TPE Métodos Numéricos Avanzados\\Modelo de Brusselator - MVMBMWM para reacciones químicas}
\date{Noviembre 2014}
\author{Federico Tedin - 53048\\Javier Fraire - 53023}

\makeindex

\begin{document}
\maketitle

\renewcommand{\abstractname}{Resumen:}
\begin{abstract}

\centering
Generación y cálculo de autovalores en forma analítica y númerica, utilizando el método QR, de la matriz correspondiente a la discretización del Jacobiano del modelo Brusselator.

\end{abstract}

\renewcommand{\abstractname}{Palabras clave:}
\begin{abstract}

\centering
\textbf{matrices ralas, autovalores, autovectores, QR, algoritmos, almacenamiento, Brusselator, Gram-Schmidt, Hessenberg}

\end{abstract}

\clearpage
\renewcommand{\contentsname}{Índice}
\tableofcontents
\clearpage

\section{Introducción}
El nombre del tema del trabajo práctico, “Brusselator wave in chemical reaction”, proviene del modelado de las concentraciones de dos sustancias químicas ubicadas en un reactor tubular.  Estas concentraciones se describen utilizando las siguientes ecuaciones diferenciales: 

\begin{equation} \label{eq:dif1}
\frac{\partial x}{\partial t} = \frac{\delta_1 \partial^2 x}{L^2  \partial^2 z} + f(x,y)
\end{equation}
\begin{equation} \label{eq:dif2}
\frac{\partial y}{\partial t} = \frac{\delta_2 \partial^2 x}{L^2  \partial^2 z} + g(x,y)
\end{equation}
donde
$$x = x(t,z)$$
$$y = y(t,z)$$
con las condiciones iniciales $x(0,z) = x_0(z), y(0,z) = y_0(z)$ y las condiciones de borde de Dirichlet $x(t,0) = x(t,1) = x^*, y(t,0) = y(t,1) = y^*$ donde $0 \leq z \leq 1$ es el espacio de coordenadas a lo largo del tubo y $t$, el tiempo.
En el modelo llamdo Brusselator se consideran:
$$f(x,y) = \alpha - (\beta + 1)x + x^2y,\:g(x,y) = \beta x - x^2y$$
Este sistema admite las soluciones estacionarias triviales $x^* = \alpha,\: y^*=\frac{\beta}{\alpha}$

Si se discretiza el intervalo $[0,1]$ usando $m$ puntos interiores con una malla de tamaño $h = \frac{1}{m+1}$, la discretización del jacobiano de \eqref{eq:dif1} y \eqref{eq:dif2} viene dada por:
$$A = 
\begin{pmatrix}
  \tau_1 T + (\beta -1) I & \alpha^2I\\
  -\beta I & \tau_2 T - \alpha^2 I 
 \end{pmatrix}$$

El artículo se divide en dos secciones principales: metodología y resultados y conclusiones. En la primera se encuentran las herramientas utilizadas y los métodos utilizados para resolver el problema planteado.  En la segunda, se detallan los resultados y se sacan conclusiones respecto a ellos.

\section{Metodología}

Para la realización del trabajo práctico se utilizó el idioma de programación Python 3.4.  Con el fin de tener un mejor control sobre los cálculos realizados, se decidió crear una clase nueva para modelar matrices matemáticas.  Esta clase permite realizar operaciones comunes entre matrices (multiplicación, suma, resta, etc.), y también armar matrices comúnmente usadas (identidad, tridiagonal, etc.).

La implementación de la clase mencionada esta separada en varios archivos.  En el archivo matrix.py se encuentra la clase \textbf{Matrix}, que modela a una matriz matematica cuadrada de tamaño $N\times N$. Esta clase incluye los métodos que describen cómo realizar operaciones entre matrices, por ejemplo, el método \emph{\_\_mul\_\_(self, other)} realiza los cálculos necesarios para multiplicar dos matrices. La clase también posee constructores que permiten crear matrices predefinidas como la identidad o la matriz nula.

Aunque la clase \textbf{Matrix} modela el comportamiento de una matriz, los valores numéricos de la matriz no se encuentran directamente almacenados allí.  En cambio, cada instancia de la clase \textbf{Matrix} posee una variable de tipo \textbf{MatrixStorage}, en donde se almacenan todos los valores.  \textbf{MatrixStorage} es una interfaz que se encuentra en el archivo \emph{matrix\_storage.py}, que describe los métodos que deberían tener clases que deseen servir como sistema de almacenamiento de los valores de la matriz.  Uno de los métodos descritos es, por ejemplo, \emph{get(i, j)}, que debe devolver el valor almacenado en la fila i, columna j.

En el trabajo práctico se experimento con dos implementaciones distintas de \textbf{MatrixStorage}.  La primera en ser creada fue \textbf{ListMatrixStorage}, en el archivo \emph{list\_matrix\_storage.py}.  Ésta clase implementa el almacenamiento de los valores como una lista de listas de valores, donde cada lista interna representa las filas de la matriz.  La segunda implementación creada fue \textbf{HashMatrixStorage}, que utiliza un la clase diccionario de Python para almacenar valores.  En éste caso, para acceder o establecer un valor de la matriz en la posición (i, j), se calcula primero el hash de la dupla que representa la posición, y a partir del valor resultante se accede al valor en el diccionario.  La ventaja de éste método es que no es necesario almacenar ceros, ya que cuando se intenta acceder a un valor (i, j) que no fue establecido en el diccionario, se devuelve automáticamente el valor 0.  El método no es perfecto ya que siempre existe la probabilidad de que ocurran colisiones: el hash de dos duplas de posiciones resultan en valores que llevan a la misma posición del diccionario.

Para crear un nuevo sistema de almacenamiento hay que implementar la clase \textbf{MatrixStorage}, e incluir la clase en \emph{matrix.py}.  Por defecto, las matrices se construyen utilizando \textbf{ListMatrixStorage}, lo cual se puede cambiar reemplazando la clase utilizada en el constructor default \emph{\_\_init\_\_(self)} de \textbf{Matrix}.

\subsection{Discretización de la matriz}
Para construir la matriz A en Python, primero se construyó cada sub-matriz.  Las matrices T e I fueron generadas con los constructores \emph{toeplitz\_tridiagonal} e \emph{identity}, respectivamente.  Una vez construidas las cuatro sub-matrices, se armó la matriz A utilizando el constructor \emph{from\_mat\_lists}.  

\subsection{Cálculo numérico de los autovalores}
Para resolver este problema se utilizó el método QR y Gram-Schmidt.

\subsubsection{Descomposición QR}
Sea $A \: \epsilon \: \mathbb{R}^{m \times n}$. Luego existen dos matrices: $Q\: \epsilon \: \mathbb{R}^{m \times }$ ortogonal y $R \: \epsilon \: \mathbb{R}^{m \times n}$ triangular superior tal que $A=Q \times R$. En el modelo de Brusselator, $A \: \epsilon \: \mathbb{R}^{n \times n}$ por lo tanto $Q \: \epsilon \: \mathbb{R}^{n \times n}$. 

\subsubsection{Gram-Schmidt}
El método de ortognalización de Gram-Schmidt es el siguiente:
\begin{enumerate}
  \item $$q_0 = \frac{\vec{v_1}}{\lVert \vec{v_1}\rVert_2}$$
  \item para cada $i \geq 2$ 
  $$\vec{e_i}=\vec{v_i}-\sum_{j=1}^{i-1} < \vec{v_i},\vec{q_j}> \times \vec{q_j}$$ 
  $$\vec{q_i} = \frac{\vec{e_i}}{\lVert \vec{e_i} \rVert}$$
  $Q$ se compone por los distintos $\vec{q_i}$ como columnas. Es decir,
  $$Q=
  \begin{pmatrix}
  \vec{q_0} & \vec{q_1} & \ldots & \vec{q_{n-1}}
 \end{pmatrix}$$
  $R$ se forma por los distintos $\lVert \vec{e_i} \rVert$ en la diagonal y el producto interno $< \vec{v_i},\vec{q_j}>$ en la fila j y la columna i. Es decir, se forma de la siguiente manera:
 $$R=
  \begin{pmatrix}
  \lVert \vec{e_0} \rVert & < \vec{v_1},\vec{q_0}> & \ldots &  < \vec{v_{n-1}},\vec{q_0}>\\
  0 & \lVert \vec{e_1} \rVert & \ldots & < \vec{v_{n-1}},\vec{q_1}>\\
  \vdots & \ddots &  & \vdots \\
  \vdots &  & \ddots & \vdots \\
  0 & \ldots & \ldots & \lVert \vec{e_{n-1}} \rVert
 \end{pmatrix}$$
\end{enumerate}
  
\subsubsection{Método QR}
\begin{equation} \label{eq:to}
T_0=A
\end{equation}
\begin{equation} \label{eq:tk}
T_{k+1}=R_k \times Q_k
\end{equation}

En el método QR los autovalores se encuentran en la diagonal de $T_{k+1}$. Un autovalor puede ser real o complejo. Sea $x_{i,i}$ un elemento de la diagonal de $T_{k+1}$, si el elemento que se encuentra por debajo es 0, entonces el autovalor es real. Es decir, si $x_{i-1,i} = 0$. En caso contrario, se obtienen dos autovalores complejos. En caso de que $x_{i-1,i} \neq 0$, se toma la siguiente matriz de $2 \times 2$:
$$
\begin{pmatrix}
x_{i,i} & x_{i, i+1} \\
x_{i+1,i} & x_{i+1,i+1}
\end{pmatrix}
$$

Luego, se obtienen los autovalores de dicha matriz de forma analítica. Dichos autovalores, serán autovalores de $A$.
Dado que en el método QR los autovalores se encuentran en la diagonal de Q, y que el algoritmo converge "de abajo hacia arriba", es decir, primero converge la parte inferior de la matriz, el algoritmo busca la convergencia de los elementos de la diagonal comenzando por el último.
Dada,  
$$Q=  
\begin{pmatrix}
x_{0,0} & x_{0,1} & \ldots & \ldots & x_{0,n-1} \\
\vdots & \ddots & & & \vdots \\
\vdots & \ldots &  x_{n-3, n-3} &  x_{n-3, n-2} & x_{n-3, n-1}\\
\vdots& \ldots & x_{n-2, n-3} & x_{n-2, n-2} & x_{n-2, n-1}\\
\vdots & \ldots & x_{n-1, n-3} & x_{n-1,n-2} & x_{n-1,n-1}

\end{pmatrix}
$$
el algoritmo comienza por $x_{n-1,n-1}$. Por lo explicado anteriormente, se busca $x_{n-1,n-2} = 0$ o $x_{n-2,n-3} =0$. Si $x_{n-1,n-2} =0$, significa que $x_{n-1,n-1}$ es un autovalor real, por lo que se agrega a la lista de autovalores obtenidos y se elimina la fila y columna a la que pertenece, ya que ya se obutvo el autovalor. Si $x_{n-2,n-3} =0$, se trata de auto valores complejos, por lo que se forma la siguiente matriz:
$$
\begin{pmatrix}
x_{n-2,n-2} & x_{n-2, n-1} \\
x_{n-1,n-2} & x_{n-1, n-1}
\end{pmatrix}
$$
Luego, se calculan los autovalores de dicha matriz y se agregan a la lista de autovalores obtenidos. Luego se eliminan las dos filas y dos columnas a las que pertenecen los elementos de la matriz ya que se obtuvieron los autovalores deseados.
Luego se repite el procedimiento con $T_{k+1} \: \epsilon  \: \mathbb{R}^{n-1 \times n-1}$ buscando la convergencia de $x_{n-2,n-3}$ o $x_{n-3,n-4}$. Esto se repite sucesivamente hasta que $T_{k+1} \: \epsilon  \: \mathbb{R}^{1 \times 1}$ o $T_{k+1} \: \epsilon  \: \mathbb{R}^{2 \times 2}$.

Como se trata de puntos flotantes, la comparación con 0 no funciona. Por lo que se debe tomar un valor para el cual se considera que un autovalor es 0. En el algoritmo, se tomo la siguiente condición para determinar si un elemento convergía a 0:
$$| x_{n-1, n-2} | < tol \times (|x_{n-1, n-1}| + |x_{n-2, n-2}|) $$
siendo $tol$ un valor de tolerancia arbitrario. En este caso se eligió, $tol = 10^{-10}$.

Para obtener los autovalores hay que llamar a la fucnión \emph{calculate\_eigenvalues\_A(m, L, delta1, delta2, alpha, beta)}. Dicha función calcula la matriz A de acuerdo a los valores recibidos. La función itera hasta que todos los autovalores hayan sido encontrados.

\subsection{Cálculo ánalitico de los autovalores}
Para obtener los autovalores de la matriz A analíticamente, se utilizó el mismo algoritmo que fue utilizado en el documento \emph{collection.ps} especificado por la cátedra.  Para calcular las raíces del polinomio formado, se utilizó el método roots de la librería NumPy.

\section{Resultados y conclusiones}
En todos los casos se utilizaron $L=0,51302$, $delta1=0,008$, $delta2=0,004$, $alpha=2$ y $beta=5,45$
\subsection{Implementaciones}
\begin{table}[H]\centering
\begin{tabular}{|c|c|c|c|}
  \hline
  m & Tamaño de la matriz & Tiempo (segundos) & Implementación \\
  \hline
  $10$ & $20 \times 20$  & $3,06$ & ListMatrixStorage\\
  \hline
  $50$ & $100 \times 100$  & $375,06$ & ListMatrixStorage\\
  \hline
  $70$ & $140 \times 140$  & $1178,64$ & ListMatrixStorage\\
  \hline
  $10$ & $20 \times 20$  & $5,56$ & HashMatrixStorage\\
  \hline
  $50$ & $100 \times 100$  & $929,85$ & HashMatrixStorage\\
  \hline
  $70$ & $140 \times 140$  & $3026,33$ & HashMatrixStorage\\
  \hline
  $10$ & $20 \times 20$  & $10,36$ & YaleMatrixStorage\\
  \hline
  $50$ & $100 \times 100$  & $3881,91$ & YaleMatrixStorage\\
  \hline
  $70$ & $140 \times 140$  & $3,06$ & YaleMatrixStorage\\
  \hline
\end{tabular}
\caption{\emph{Tiempo de cálculo de autovalores de las distintas implementaciones.}}
\end{table}
Como se puede observar en la tabla para matrices  $100 \times 100$ y para matrices más chicas  tardó un tiempo razonable, pero para matrices $140 \times 140$ tardó más de 15 minutos. Por lo tanto se recomienda como tamaño máximo de matriz utilizar 140 x 140. También se puede observar que el método no es eficiente ya que tarda mucho.

Las matrices con las que opera el programa son ralas. Como se mencionó anteriormente se realizó una implementación de esta matrices llamada \textbf{HashMatrixStorage} que utiliza un diccionario para guardar los elementos. Esto permite ahorrar memoria y procesar menos datos, pero como se puede observar en la tabla 1, en la práctica probó ser más lento. Tardando el cuádruple que utilizando la implementación con listas. Tambien se realizó una implementacion llamada \textbf{YaleMatrixStorage} la cúal también ahorraba espacio y también probo ser más lenta. Además, se intentó realizar una variacíon a la multiplicación llamada \emph{multiply\_quick\_values}. Esta, en lugar de multiplicar las filas y columnas completas, se obtenían las filas y columnas solo con los elementos que fueran distintos de 0. De esta manera, se evita iterar sobre una gran cantidad de números, "salteando" los 0. Está implementación también resulto ser más lenta.

El orden de complejidad del algoritmo en cada iteración es $O(n^3)$. Sabemos de la existencia de otros algoritmos que operan mejor, pero no pudimos realizar una implementación que funcionara correctamente. Por ejemplo utilizando Hessenberg se podría reducir la complejidad de cada iteración a $O(n)$. 

En una primera iteración del trabajo, en lugar de utilizar $T_{k+1}=R_k \times Q_k$, se había utilizado $T_{k+1}= Q_k^T \times T_{k} \times Q_k$. Trasponer la matriz es una operación más costosa que calcular R por lo que al cambiar la forma de calcular $T_{k+1}$, se mejoró la velocidad del algoritmo de forma significativa. La siguiente tabla ilustra lo mencionado:

\begin{table}[H]\centering
\begin{tabular}{|c|c|c|c|}
  \hline
  m & Tamaño de la matriz & Tiempo (segundos) & $T_{k+1}$ \\
  \hline
  $5$ & $10 \times 10$  & $0,745$ & $Q_k^T \times T_{k}$\\
  \hline
  $10$ & $20 \times 20$  & $4,16$ & $Q_k^T \times T_{k}$\\
  \hline
  $15$ & $30 \times 30$  & $11,06$ & $Q_k^T \times T_{k}$\\
  \hline
  $20$ & $40 \times 40$  & $26,047$ & $Q_k^T \times T_{k}$\\
  \hline
  $5$ & $10 \times 10$  & $0,496$ & $R_k \times Q_k$\\
  \hline
  $10$ & $20 \times 20$  & $3,06$ & $R_k \times Q_k$\\
  \hline
  $15$ & $30 \times 30$  & $7,56$ & $R_k \times Q_k$\\
  \hline
  $20$ & $40 \times 40$  & $18,79$ & $R_k \times Q_k$\\
  \hline
\end{tabular}
\caption{\emph{Tiempo de cálculo de autovalores utilizando distintas formas de calcuar $T_{k+1}$.}}
\end{table}

\subsection{Comparación de resultados númericos y análiticos}
\begin{table}[H]\centering
\begin{tabular}{|c|c|c|c|}
  \hline
  m & Tamaño de la matriz & Tiempo (segundos) & Cálculo \\
  \hline
  $5$ & $10 \times 10$  & $0,496$ & Númerico\\
  \hline
  $10$ & $20 \times 20$  & $3,06$ & Númerico\\
  \hline
  $15$ & $30 \times 30$  & $7,56$ & Númerico\\
  \hline
  $20$ & $40 \times 40$  & $18,79$ & Númerico\\
  \hline
  $5$ & $10 \times 10$  & $0,00073$ & Analítico\\
  \hline
  $10$ & $20 \times 20$  & $0.012$ & Analítico\\
  \hline
  $15$ & $30 \times 30$  & $0.016$ & Analítico\\
  \hline
  $20$ & $40 \times 40$  & $0,0021$ & Analítico\\
  \hline
\end{tabular}
\caption{\emph{Tiempo de cálculo de autovalores utilizando cálculos analíticos y númericos}}
\end{table}
Como se puede observar en la Tabla 3, la diferencia de tiempo entre el cálculo análitico y el númerico es importante. Siendo el cálculo númerico varias veces más rapido. Para $m=5$, los autovalores son exactamente iguales, pero para $m=20$, por ejemplo, difieren. La siguiente tabla compara unos pocos autovalores para $m=20$.

\begin{table}[H]\centering
\begin{tabular}{|c|c|}
  \hline
  Cálculo númerico & Cálculo analítico \\
  \hline
  $0,000419372599688+2,13923811584i$ & 						$0,000419372599686+2,13923811584i$\\
    \hline
  $-25,65780914206229$ & $-25.6578091421$\\
    \hline
  $-30.876122264558283$ & $-30.8761222646$\\
  \hline
  $-9.82859178427-4.5865530384i$ & $-9.82859178427-4.5865530384i$\\
  \hline
  $47.5814180808$ & $-47.5814180808$\\
  \hline
\end{tabular}
\caption{\emph{Algunos autovalores para $m=5$ utilizando cálculo númerico y analítico}}
\end{table}
Conviene utilizar el cálculo analítico ya que es significativamente más rápido.

\clearpage

\renewcommand\refname{Referencias}
\begin{thebibliography}{9}
\bibitem{collection}
 Zhaojun Baiy, David Day, James Demmel and Jack Dongarra \emph{A Test Matrix Collection for Non-Hermitian Eigenvalue Problems}. \url{http://www.cs.ucdavis.edu/~bai/NEP/document/collection.ps}
\bibitem{matMarket}
\url{http://math.nist.gov/MatrixMarket/data/NEP/mvmbwm/mvmbwm.html} (Septiembre/Octubre 2014)
\bibitem{people}
\url{http://people.inf.ethz.ch/arbenz/ewp/Lnotes/chapter3.pdf}
(Septiembre/Octubre 2014)
\end{thebibliography}

\end{document}